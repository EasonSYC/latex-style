\documentclass{article}
\usepackage{eason}

\usepackage{pdfpages}

\usepackage[a4paper, margin=1in]{geometry}

\usepackage{lipsum}

\title{Test Document}
\author{Eason Shao}
\date{\today}

\begin{document}
\maketitle

\section{Symbols}
\subsection{Misc}
\[
    \epsilon, n\inverse, \LHS, \RHS, \theta\degree
\]

\subsection{Sets}
\[
    \NN, \ZZ, \QQ, \RR, \CC
\]
\[
    \PP, \FF, \QQ\multiplicative, \size\NN, \emptyset
\]

\subsection{Functions}
\[
    \img, \ker
\]
\[
    \sign, \identity, \indicator, \dd
\]

\subsubsection{Trigonometric and Hyperbolic Functions}
\[
    \sin, \cos, \tan, \cot, \sec, \csc
\]
\[
    \arcsin, \arccos, \arctan, \arccot, \arcsec, \arccsc
\]
\[
    \sinh, \cosh, \tanh, \coth, \sech, \csch
\]
\[
    \arsinh, \arcosh, \artanh, \arcoth, \arsech, \arcsch
\]

\subsubsection{UK Notation}
\[
    \cosec, \cosech, \arccosec, \arcosech
\]

\subsubsection{Exponential and Logarithmic Functions}
\[
    \exp, \log, \ln, \lg, \lb
\]

\subsection{Number Theory}
\[
    \totient, \gcd, \lcm, \max, \min
\]
\[
    7 \modulo 2, 2 \divides 6, 2 \notdivides 7
\]

\subsection{Group Theory}
\[
    \ord, \Isom, \Sym, \Fix, \orb, \stab, \ccl
\]
\[
    \trivialgp, \actson, \subgroup, \normalsub, \isomorphic
\]

\subsection{Analysis}
\[
    \LUB, \supremum, \sup, \GLB, \infimum, \inf
\]
\[
    \limsup, \liminf, \lim
\]

\subsubsection{Infinity}
\[
    \infinity, \plusinfinity, \minusinfinity
\]

\subsubsection{Differentiation}
\[
    \Diff x, \DiffFrac{y}{x}, \DiffOp{x}
\]
\[
    \DiffN{2} x, \DiffNFrac{y}{x}{2}, \DiffNOp{x}{2}
\]
\[
    \Partial x, \PartialFrac{y}{x}, \PartialOp{x}
\]
\[
    \PartialN{2} x, \PartialNFrac{y}{x}{2}, \PartialNOp{x}{2}
\]

\subsection{Probability}
\[
    \Prob, \Expt, \Var, \Cov, \Corr
\]

\subsubsection{Distribution}
\[
    \Binomial, \Poisson, \Normal, \Exponential, \Geometric, \Uniform
\]

\subsection{Complex Numbers}
\[
    \arg, \im, \re, \conjugate{z}
\]

\subsection{Linear Algebra}
\[
    \det, \tr, \adj, \nul, \rank, \spn
\]

\subsubsection{Matrices}
\[
    \vm{M}, \identityM, \zeroM, \vm{M}\transpose, \vm{M}\hermitian
\]

\subsubsection{Matrix Groups}
\[
    \GL, \SL, \Or, \SO, \U, \SU, \PGL, \PSL
\]

\subsubsection{Basis Vectors}
\[
    \ihat, \jhat, \khat
\]

\subsection{Paired Delimiters}
\[
    \para{\frac{a}{b}}, \brac{\frac{a}{b}}, \brce{\frac{a}{b}}
\]
\[
    \ceil{\frac{a}{b}}, \floor{\frac{a}{b}}, \abs{\frac{a}{b}}, \ang{\frac{a}{b}}
\]
\[
    \set{x \in \RR}{x = \frac{a}{b}}
\]

\section{Theorems}
\begin{definition}[Some Definition]
    \label{def}%
    This is a \emph{definition}. \lipsum[1]
\end{definition}

\begin{theorem}[Very Important Theorem]
    \label{thm}%
    This is a very important theorem. \lipsum[2]
\end{theorem}

\begin{proof}
    A proof. \lipsum[3]
\end{proof}

\begin{proof}[of \autoref{thm}]
    A proof of \autoref{thm}. \lipsum[4]
\end{proof}

\begin{examples}
    Some examples of the theorem. \lipsum[5]
\end{examples}

\begin{notation}
    The previous theorem allows us to abuse notation. \lipsum[6]
\end{notation}

\begin{corollary}[Obvious Corollary]
    \label{cor}%
    A corollary. \lipsum[7]
\end{corollary}

\begin{example}[An example]
    An example. \lipsum[8]
\end{example}

\begin{lemma}[Some Lemma]
    \label{lem}%
    Some lemma. \lipsum[9]
\end{lemma}

\begin{claim}[Some Claim]
    \label{cla}%
    Some claim. \lipsum[10]
\end{claim}

\begin{remark}
    This is a remark on the claim. \lipsum[11]
\end{remark}

\begin{proposition}[Some Proposition]
    \label{pro}%
    A proposition. \lipsum[12]
\end{proposition}

\begin{remarks}
    Some remarks on this proposition. \lipsum[13]
\end{remarks}

\begin{problem}[Some Problem]
    \label{prm}%
    A problem. \lipsum[14]
\end{problem}

\begin{solution}
    A solution. \lipsum[15]
\end{solution}

\begin{solution}[of \autoref{prm}]
    A solution of \autoref{prm}. \lipsum[16]
\end{solution}

Referencing works as well, like \autoref{def}, \autoref{thm}, \autoref{cor}, \autoref{lem}, \autoref{cla}, \autoref{pro}, and \autoref{prm}.

\end{document}